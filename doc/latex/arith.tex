\section{Arithmetic Functions}
\markright{\arabic{section}. Arithmetic}
\subsection{Arithmetic Constants}
\begin{refdesc}
\constdesc{most-positive-fixnum}{\#x1fffffff=536,870,911}
\constdesc{most-negative-fixnum}{-\#x20000000= -536,870,912}
\constdesc{short-float-epsilon}{
A floating point number on machines with IEEE floating-point format is
represented by 21 bit mantissa with 1 bit sign and 7 bit exponent with
1 bit sign.
Therefore, floating point epsilon is $2^{-21}= 4.768368 \times 10^{-7}$.}
\constdesc{single-float-epsilon}{same as {\bf short-float-epsilon}, $2^{-21}$.}
\constdesc{long-float-epsilon}{same as {\bf short-float-epsilon} since
there is no double or long float. $2^{-21}$.}
\constdesc{pi}{$\pi$, actually 3.14159203, not 3.14159265.}
\constdesc{2pi}{$2\times \pi$}
\constdesc{pi/2}{$\pi/2$}
\constdesc{-pi}{-3.14159203}
\constdesc{-2pi}{$-2\times \pi$}
\constdesc{-pi/2}{$\pi/2$}
\end{refdesc}

\subsection{Arithmetic Predicates}
\begin{refdesc}
\funcdesc{numberp}{object}{
T if {\em object} is number, namely integer or float.
Note that characters are also represented by numbers.}
\funcdesc{integerp}{object}{
T if {\em object} is an integer number.
A float can be converted to an integer by
{\bf round, trunc} and {\bf ceiling} functions.}
\funcdesc{floatp}{object}{T if {\em object} is a floating-point number.
An integer can be converted to a float by the {\bf float} function.}

\funcdesc{zerop}{number}{T if the number is integer zero or float 0.0.}
\funcdesc{plusp}{number}{equivalent to ($>$ number 0).}
\funcdesc{minusp}{number}{equivalent to ($<$ number 0).}

\funcdesc{oddp}{integer}{
The argument must be an integer.
T if {\em integer} is odd.}

\funcdesc{evenp}{integer}{
The argument must be an integer.
T if {\em integer} is an even number.}

\funcdesc{/=}{n1 n2 \&rest more-numbers}{
Both {\em n1}, {\em n2} and all elements of {\em more-numbers} must be numbers.
T if no two of its arguments are numerically equal, NIL otherwise.}

\funcdesc{=}{num1 num2 \&rest more-numbers}{
Both {\em n1} and {\em n2} and all elements of {\em more-numbers} must be numbers.
T if {\em n1}, {\em n2} and all elements of {\em more-numbers} are the same in value, NIL otherwise.}

\funcdesc{$>$}{num1 num2 \&rest more-numbers}{
Both {\em n1} and {\em n2} and all elements of {\em more-numbers} must be numbers.
T if {\em n1}, {\em n2} and all elements of {\em more-numbers} are in monotonically decreasing order, NIL otherwise.
For numerical comparisons with tolerance, use functions prefixed
by {\bf eps} as described in the section \ref{Geometry}.}

\funcdesc{$<$}{num1 num2 \&rest more-numbers}{
Both {\em n1} and {\em n2} and all elements of {\em more-numbers} must be numbers.
T if {\em n1}, {\em n2} and all elements of {\em more-numbers} are in monotonically increasing order, NIL otherwise.
For numerical comparisons with tolerance, use functions prefixed
by {\bf eps} as described in the section \ref{Geometry}.}

\funcdesc{$>=$}{num1 num2 \&rest more-numbers}{
Both {\em n1} and {\em n2} and all elements of {\em more-numbers} must be numbers.
T if {\em n1}, {\em n2} and all elements of {\em more-numbers} are in monotonically nonincreasing order, NIL otherwise.
For numerical comparisons with tolerance, use functions prefixed
by {\bf eps} as described in the section \ref{Geometry}.}

\funcdesc{$<=$}{num1 num2 \&rest more-numbers}{
Both {\em n1} and {\em n2} and all elements of {\em more-numbers} must be numbers.
T if {\em n1}, {\em n2} and all elements of {\em more-numbers} are in monotonically nondecreasing order, NIL otherwise.
For numerical comparisons with tolerance, use functions prefixed
by {\bf eps} as described in the section \ref{Geometry}.}
\end{refdesc}

\subsection{Integer and Bit-Wise Operations}
Following functions request arguments to be integers.

\begin{refdesc}
\funcdesc{mod}{dividend divisor}{
returns remainder when {\em dividend} is divided by {\em divisor}.
{\tt (mod 6 5)=1, (mod -6 5)=-1, (mod 6 -5)=1, (mod -6 -5)=-1}.}

\funcdesc{1-}{integer}{
The compiler assumes the argument to be an integer.
$integer-1$ is returned.}

\funcdesc{1+}{integer}{
Arguments to 1+ and 1- must be an integer.
$integer+1$ is returned.}
\funcdesc{logand}{\&rest integers}{bitwise-and of {\em integers}.}
\funcdesc{logior}{\&rest integers}{bitwise-inclusive-or of {\em integers}.}
\funcdesc{logxor}{\&rest integers}{bitwise-exclusive-or of {\em integers}.}
\funcdesc{logeqv}{\&rest integers}{
{\bf logeqv} is equivalent to (lognot (logxor ...)).}
\funcdesc{lognand}{\&rest integers}{bitwise-nand of {\em integers}.}
\funcdesc{lognor}{\&rest integers}{bitwise-nor of {\em integers}.}
\funcdesc{lognot}{integer}{ bit reverse of {\em integer}.}
\funcdesc{logtest}{integer1 integer2}{
T if (logand {\em integer1 integer2}) is not zero.}

\funcdesc{logbitp}{index integer}{
T if {\em index}th bit of {\em integer} (counted from the LSB) is 1,
otherwise NIL.}

\funcdesc{ash}{integer count}{
Arithmetic Shift Left.
If {\em count} is positive, shift direction is left,
and if {\em count} is negative,
{\em integer} is shifted to right by abs({\em count}) bits.}

\funcdesc{ldb}{target position width}{
LoaD Byte.
Byte specifier for {\bf ldb} and {\bf dpb} does not exist in EusLisp.
Use a pair of integers instead.
The field of {\em width} bits at {\em position} within {\em target}
is extracted. For example, {\tt (ldb \#x1234 4 4)} is 3.}

\funcdesc{dpb}{value integer position width}{
DePosit byte.
{\em Width} bits of {\em value} is put in {\em integer}
at {\em position}th bits from LSB.}

\end{refdesc}


\subsection{Generic Number Functions}
\begin{refdesc}
\funcdesc{+}{\&rest numbers}{returns the sum of {\em numbers}.}
\funcdesc{-}{num \&rest more-numbers}{
If {\em more-numbers} are given, they are subtracted from {\em num}.
Otherwise, {\em num} is negated.}
\funcdesc{*}{\&rest numbers}{returns the product of {\em numbers}.}
\funcdesc{/}{num1 num2 \&rest more-numbers}{
{\em num1} is divided by {\em num2} and {\em more-numbers}.
The result is an integer if all the arguments are integers,
and an float if at least one of the arguments is a float.}
\funcdesc{abs}{number}{returns absolute number.}
\funcdesc{round}{number}{
rounds to the nearest integer.
{\tt (round 1.5)=2, (round -1.5)=2}.}

\funcdesc{floor}{number}{
rounds to the nearest smaller integer.
{\tt (floor 1.5)=1, (floor -1.5)=-2}.}

\funcdesc{ceiling}{number}{
rounds to the nearest larger integer.
{\tt (ceiling 1.5)=2, (ceiling -1.5)=-1}.}

\funcdesc{truncate}{number}{
rounds to the absolutely smaller and nearest integer.
{\tt (truncate 1.5)=1, (truncate -1.5)=-1}.}

\funcdesc{float}{number}{
returns floating-point representation of {\em number}.}

\funcdesc{max}{\&rest numbers}{
finds the maximum value among {\em numbers}.}

\funcdesc{min}{\&rest numbers}{
finds the minimum number in {\em numbers}.}

\funcdesc{random}{range \&optional (randstate *random-state*)}{
Returns a random number between 0 or 0.0 and {\em range}.
If {\em range} is an integer,
the result is truncated to an integer.
Otherwise, a floating value is returned.
Optional {\em randstate} can be specified
to get predictable random number sequence.
There is no special data type for random-state,
and it is represented with an integer vector of two elements.
}

\macrodesc{incf}{variable \&optional (increment 1)}{
{\em variable} is a generalized variable.
The value of {\em variable} is incremented by {\em increment},
and it is set back to {\em variable}.}
\macrodesc{decf}{variable \&optional decrement}{
{\em variable} is a generalized variable.
The value of {\em variable} is decremented by {\em decrement},
and it is set back to {\em variable}.}

\funcdesc{reduce}{func seq}{
combines all the elements in {\em seq} using the binary operator {\em func}.
For an example, {\tt (reduce \#'expt '(2 3 4)) = (expt (expt 2 3) 4)=4096}.}

\funcdesc{rad2deg}{radian}{Radian value is converted to degree notation.
\#R does the same thing at read time.
Note that the official representation of angle in EusLisp is radian
and every EusLisp function that accepts angle argument
requests it to be represented by radian.}
\funcdesc{deg2rad}{degree}{Conversion from degree to radian.
Also accomplished by \#D reader's dispatch macros.}
\end{refdesc}

\subsection{Trigonometric and Related Functions}
\begin{refdesc}
\funcdesc{sin}{theta}{{\em theta} is a float representing angle by radian.
returns $sin(theta)$.}
\funcdesc{cos}{theta}{{\em theta} is a float representing angle by radian.
returns $cos(theta)$.}
\funcdesc{tan}{theta}{{\em theta} is a float representing angle by radian.
returns $tan(theta)$.}
\funcdesc{sinh}{x}{ hyperbolic sine, that is,
$\frac{e^{x}-e^{-x}}{2}$.}
\funcdesc{cosh}{x}{ hyperbolic cosine, that is,
$\frac{e^{x}+e^{-x}}{2}$.}
\funcdesc{tanh}{x}{ hyperbolic tangent, that is,
$\frac{e^{x}+e^{-x}}{e^{x}-e^{-x}}$.}
\funcdesc{asin}{number}{arc sine of {\em number}.}
\funcdesc{acos}{number}{arc cosine of {\em number}.}
\funcdesc{atan}{y \&optional x}{
When {\bf atan} is called with one argument, its arctangent is calculated.
When called with two arguments, $atan(y/x)$ is returned.}
\funcdesc{asinh}{x}{hyperbolic arc sine.}
\funcdesc{acosh}{x}{hyperbolic arc cosine.}
\funcdesc{atanh}{x}{hyperbolic arc tangent.}

\funcdesc{sqrt}{number}{returns square root of {\em number}.}

\funcdesc{log}{number}{returns natural logarithm of {\em number}.}

\funcdesc{exp}{x}{returns exponential, $e^{x}$.}

\funcdesc{expt}{a x}{
returns {\em x}th power to {\em a}.}
\end{refdesc}

\newpage
