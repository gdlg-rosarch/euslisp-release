\section{Introduction}
\markright{\arabic{section}. Introduction}
% EusLispは、知能ロボットの研究を目的とした言語で、Common Lispと
% 対象指向型プログラミングに基礎を置いている。ロボットの研究では、
% センサデータの処理、環境認識、障害物回避動作計画、作業計画などが
% 重要なテーマとして取り上げられるが、これらに共通するのは、ロボットと
% 外界の3次元幾何モデルである。EusLispを開発する動機となったのは、
% 記号処理システムから簡単に使えて拡張性の高いソリッドモデラーへの
% 要求である。いくつかの既存のソリッドモデラーを調べてわかったことは、
% その実現にとって最も重要な機能は、数値演算ではなくモデル要素の
% トポロジーを表現、管理するためのリスト処理能力であった。もちろん、
% 幾何演算も重要ではあるが、ベクタ、マトリクスを操作する関数を組み込めば
% 十分であることがわかった。
% こうして、ソリッドモデラーはLispの上に実現されるべきであると言う
% 基本方針が得られた。さらに、ソリッドモデラーは、3次元物体モデルを
% 定義し、動きをシミュレートし、物体の相互関係を表現し、グラフィックス
% 表示する機能を提供するが、先に述べた各種のロボット問題と結合されなけ
% れば意味がない。また、ロボットを完成されたシステムとして実現するには、
% これらのロボット問題を解くモジュールが効果的に統合されなければならない。
% EusLispは、この統合の枠組をオブジェクト指向に求めた。オブジェクト指向は、
% モジュラープログラミングを促進し、継承機能により既存の機能を段階的に
% 拡張することが容易になる。実際、上記のソリッドモデラーは、物体、面、
% 稜線などの物理的実体の振舞いをクラスに定義し、ロボット問題に依存する
% 機能は、これらをサブクラスに拡張することで効率よく発展させられる。
% これは、ソフトウェア資源の再利用にもつながる。
% こうしてEusLispは、対象指向とCommonLispをベースとして3次元幾何モデラー
% を実現し、複数ロボットの協調動作に必要なタスク間通信機能、マンマシン
% インタフェースに重要なウィンドウ、グラフィックス、複合的プログラミングに
% 必要な他言語インタフェース等を備え、さまざまなロボット問題への適用を
% 可能にしたプログラミングシステムとして構築された。この他、メモリ管理にも
% 工夫を凝らし、メモリ容量以外に生じる領域の大きさに関する制約を極力
% 排除し、ごみ集めが効率的に行なわれ、ユーザがメモリ管理に関するパラ
% メータを操作する必要がないように努めた。
% このreference manualは、EusLisp BasicsとExtensionsに分かれ、
% 前者がCommon Lispの機能とオブジェクト指向型プログラミングを、後者が
% 幾何モデル、ロボットモデル、ウィンドウ、画像処理など、よりロボット
% 応用に近い部分を扱っている。


EusLisp is an integrated programming system 
for the research on intelligent robots
based on Common Lisp and Object-Oriented programming.
The principal subjects
in the field of robotics research are  sensory data processing,
visual environment recognition, collision avoiding motion planning,
and task planning.
In either problem, three dimensional shape models of robots and
environment play crucial roles.
A motivation to the development of EusLisp was a demand for an extensible
solid modeler that can easily be made use of from higher level symbolic
processing system.
Investigations into traditional solid modelers proved that the vital
requirement for their implementation language was the list processing
capability to represent and manage topology among model components.
Numerical computation power was also important, but locality of geometric
computation suggested the provision of vector/matrix functions as
built-ins would greatly ease programming.

Thus the primary decision to build a solid modeler in a Lisp
equipped with a geometric computation package was obtained.
Although a solid modeler provides facilities to define shapes of 3D
objects, to simulate their behaviors, and to display them graphically,
its applications are limited until it is incorporated in robot modules
mentioned above. These modules also need to be tightly interconnected
to achieve fully integrated robot systems.
EusLisp sought for the framework of this integration
in object-oriented programming (OOP).
While OOP promotes modular programming, it facilitates incremental
extension of existing functions by using inheritance of classes.
In fact, components in the solid modeler, such as bodies, faces, and edges,
can orderly be inplemented by extending one of the most basic class
{\em coordinates}.
These components may have further subclasses to provide individual functions
for particular robot applications.

Based upon these considerations, EusLisp has been developped as an
object-oriented Lisp which implements an extensible solid modeler\cite{Eus7}.
Other features include intertask communication needed for the cooperative
task coordination, graphics facilities on X-window for 
visual user interface, and foreign language interface to support
mixed language programming.

In the implementation of the language, two performance-effective techniques
were invented in type discrimination and memory management
\cite{Eus4,Eus5,Eus6}.
The new type discrimination method guarantees 
constant-time discrimination between
types in tree structured hiearchy without regard to the depth of trees.
Heap memory is managed in Fibonacci buddy method, which improves
memory efficiency without sacrificing runtime or garbage-collection
performance.

This reference manual describes EusLisp version 7.27 in two parts,
{\em  EusLisp Basics and EusLisp Extensions}.
The first part describes Common Lisp features and object-oriented 
programming. Since a number of literatures are available on both topics,
the first part is rather indifferent except EusLisp's specific
features as described in {\em Interprocess Communication and Network},
{\em Toplevel Interaction}, {\em Disk Save}, etc.
Beginners of EusLisp are advised to get familiar with
Common Lisp and object oriented programming in other ways
\cite{CLtL,CLOS:Keene}.
The second part deals with features more related with robot applications,
such as
{\em Geometric Modelling}, {\em Image Processing}, {\em Manipulator Model}
and so on.
Unfortunately, 
the descriptions in this part may become incomplete or inaccurate
because of EusLisp's rapid evolution.
The update information is available via euslisp mailing list
as mentioned in section \ref{License}.

\subsection{EusLisp's Object-Oriented Programming}
Unlike other Lisp-based object-oriented programming languages like
CLOS \cite{CLOS:Keene},
EusLisp is a Lisp system built on the basis of object-orientation.
In the former approach, Lisp is used as an
implementation language for the object-oriented programming,
and there is apparent distinction between system defined objects
and user defined objects, since system data types do not have 
corresponding classes.
On the other hand, every data structure in EusLisp except number is 
represented by an object, and there is no inherent difference between
built-in data types, such as {\tt cons} and {\tt symbols},
and user defined classes.
This implies that even the system built-in data types
can be extended (inherited) by user-defined classes.
Also, when a user defines his own class as a subclass of a built-in class,
he can use built-in methods and functions for the new class,
and the amount of description for a new program can be reduced.
For example, you may extend the {\tt cons} class to have extra field other
than {\tt car} and {\tt cdr} to define queues, trees, stacks, etc.
Even for these instances, built-in functions for
built-in cons are also applicable without any loss of efficiency,
since those functions recognize type hierarchy in a constant time.
Thus, EusLisp makes all the system built-in facilities open to programmers
in the form of extensible data types.
This uniformity is also beneficial to the implementation of EusLisp,
because, after defining a few kernel functions
such as {\bf defclass}, {\bf send}, and {\bf instantiate},
in the implementation language,
most of house-keeping functions to access the internal structure of built-in
data types can be coded in EusLisp itself.
This has much improved the reliability and maintainability of EusLisp.

\subsection{Features}

\begin{description}
\item [object-oriented programming]
EusLisp provides single-inheritance Object-Oriented programming.
All data types except numbers are represented by objects whose
behaviors are defined in their classes.
\item [Common Lisp] EusLisp follows the specifications of Common Lisp
described in \cite{CLtL} and \cite{CLtL2}
as long as they are consistent with EusLisp's goal and object-orientation.
See next subsection for incompatibilities.
\item [compiler]
EusLisp's compiler can boost the execution  5 to 30 
times as fast as the interpreted execution.
The compiler keeps the same semantics as the interpreter.
\item[memory management] Fibonacci buddy method,
which is memory efficient, GC efficient, and robust,
is used for the memory management.
EusLisp can run on machines with relatively modest amount of memory.
Users are free from the optimization of page allocation for each
type of data.
\item [geometric primitives]
Since numbers are always represented as immediate data,
no garbage is generated by numeric computation.
A number of geometric functions for arbitrary-sized  vectors and matrices
are provided as built-in functions.
\item [geometric modeler]
Solid models can be defined from primitive bodies using CSG set operations.
Mass properties, interference checking, contact detection, and so on,
are available.
\item [graphics]
Hidden-line eliminated drawing and hidden-surface eliminated rendering
are available.
Postscript output to idraw can be generated.
\item [image processing]
Edge based image processing facility is provided.
\item [manipulator model]
6 D.O.F.s robot manipulator can easily be modeled.
\item [Xwindow interface]
Three levels of Xwindow interface, the Xlib foreign functions, 
the Xlib classes and the original XToolKit classes are provided.
\item [foreign-language interface]
Functions written in C or other languages can be linked into EusLisp.
Bidirectional call between EusLisp and other language are supported.
Functions in libraries like LINPACK become available through this interface.
Call-back functions in X toolkits can be defined in Lisp.
\item [unix binding]
Most of unix system calls and unix library functions are assorted as Lisp
functions. Signal handling and asynchronous I/O are also possible.
\item [multithread] multithread programming, which enables multiple
contexts sharing global data, is available on Solaris 2 operating system.
Multithread facilitates asynchronous programming and improves real-time
response\cite{MTEus1,MTEus2}.
If EusLisp runs on multi-processor machines, it can utilize 
parallel processors' higher computating power.
%\item[disk save]
%EusLisp process image can be saved in a disk file to ease initial loading
%or to produce a complete application.
\end{description}

\subsection{Compatibility with Common Lisp}

Common Lisp has become the well-documented and widely-available standard Lisp
\cite{CLtL,CLtL2}.
Although EusLisp has introduced lots of Common Lisp features
such as variable scoping rules, packages, sequences, generalized variables,
blocks, structures, keyword parameters, etc.,
incompatibilities still remain.
Here is a list of missing features:

\begin{enumerate}
\item multiple values:
      multiple-value-call,multiple-value-prog1, etc.
\item some of data types:
      complex number, bignum, ratio, character and deftype
\item some of special forms:
      progv, compiler-let,macrolet
\end{enumerate}

Following features are incomplete:
\begin{enumerate}
\setcounter{enumi}{3}
\item  closure -- only valid for dynamic extent
%\item  package --  no shadowing-list
\item  declare,proclaim -- inline and ignore are unrecognized
\end{enumerate}

\subsection{Revision History}
\begin{description}
\item[1986] The first version of EusLisp ran on Unix-System5/Ustation-E20.
Fibonacci buddy memory management, simple compiler generating M68020
assembly code, and vector/matrix functions were tested.
\item[1987] The new fast type checking method is implemented.
The foreign language interface and the SunView interface were incorporated.
\item[1988] The compiler was changed to generate C programs as
intermediate code. Since the compiler became processor independent,
EusLisp was ported on Ultrix/VAX8800 and on SunOS3.5/Sun3 and /Sun4 .
IPC facility using socket streams was added.
The solid modeler was implemented.
Lots of Common Lisp features such as keyword parameters, 
labeled print format to handle recursive data objects,
generic sequence functions,
readtables, tagbody, go, flet, and labels special forms, etc.,
were added.
\item[1989] The Xlib interface was introduced.
\% read macro to read C-like mathematical expressions was made.
manipulator class is defined.
\item[1990] The XView interface was written by M.Inaba.
Ray tracer was written.
Solid modeler was modified to keep CSG operation history.
Asynchronous I/O was added.
\item[1991] The motion constraint program was written by H.Hirukawa.
Ported to DEC station.
Coordinates class changed to handle both 2D and 3D coordinate systems.
Body composition functions were enhanced to handle contacting objects.
CSG operation for contacting objects.
The package system became compatible with Common Lisp.
\item[1992]
{\bf Face+} and {\bf face*} for union and intersection of two coplanar faces
were added.
Image processing facility was added. The first completed reference manual
was printed and delivered.
\item[1993] EusLisp was stable.
\item[1994] Ported to Solaris 2. Multi-context implementation using
Solaris's multithread facility. XToolKit is built. Multi robot simulator,
MARS was written by Dr. Kuniyoshi. EusLisp organized session at RSJ 94,
in Fukuoka.
\item[1995] The second version of the reference manual is published.
\item[2010] Version 9.00 is releaced, The licence is changed to BSD.
\item[2011] Add Darwin OS Support, Add model files.
\item[2013] Add Cygwin 64 Bit support, expand MXSTACK from 65536 to 8388608, KEYWORDPARAMETERLIMIT from 32 to 128.
\item[2014] Use UTF-8 for documents, Version 9.10 is releaced.
\item[2015] more error check on min/max, support arbitrary length for vplus, more quiet for non-ttyp mode, Version 9.11 is releaced.
\item[2015] Version 9.12 is released, support ARM
            Version 9.13 is released, support class documentation
            Version 9.14 is released, fix assert API. Now message is optional (defmacro assert (pred \&optional message)
            Version 9.15 is released, fix char comparison function (previous version retuns opossite result), support multiple argument at function /=,  add url encode feature (escape-url function), support microsecond add/subtract in interval-time class
            Version 9.16 is released, added make-random-state, fixed bug in lib/llib/unittest.l
\end{description}

\subsection{Installation}
The installation procedure is described in {\tt README}.
The installation directory, which is assumed to be {\tt "/usr/local/eus/"},
should be set to the global variable 
{\tt *eusdir*}, since this location is referenced
by {\bf load} and the compiler.

Subdirectories in  {\tt *eusdir*} are described in table \ref{Directories}.
Among these, 
{\tt c/, l/, comp/, geo/, clib/,} and {\tt xwindow} contain essential
files to make eus and eusx. Others are optional libraries, demonstration
programs and contributions from users.

\begin{table}
\begin{center}
{\footnotesize
\begin{tabular}{|l | l|}\hline 
FILES &  this document \\
README &  a brief guide to lisence, installation and sample run\\
VERSION &  EUSLisp version number\\
bin &  executables (eus, euscomp and eusx) \\
c/ &  EusLisp kernel written in C\\
l/ &  kernel functions written in EusLisp\\
comp/ &  EusLisp compiler written in EusLisp\\
clib/ &  library functions written in C\\
doc/ &  documentation (latex and jlatex sources and memos)\\
geo/ &  geometric and graphic programs\\
lib/ & shared libraries (.so) and start-up files\\
llib/ &  Lisp library \\
llib2/ &  secondary Lisp library developed at UTYO\\
xwindow/ &  X11 interface\\
makefile@ &  symbolic link to one of makefile.sun[34]os[34],.vax, etc.\\
pprolog/ &  tiny prolog interpreter\\
xview/ &  xview tool kit interface\\
tool/ & \\
vxworks/ & interface with VxWorks real-time OS\\
robot/ & robot models and simulators\\
vision/ &  image processing programs\\
contact/ & motion constraint solver by H.Hirukawa
\cite{Hirukawa:1991a,Hirukawa:1991b,Hirukawa:1991c}\\
demo/ &  demonstrative programs\\
bench/ &  benchmark programs\\ \hline
\end{tabular}
}
\end{center}
\caption{\label{Directories}Directories in {\tt *eusdir*}}
\end{table}

\subsection{\label{License}License}

EusLisp is distributed under the following BSD License.

\begin{verbatim}
Copyright (c) 1984-2001, National Institute of Advanced Industrial Science
 and Technology (AIST)
All rights reserved.

Redistribution and use in source and binary forms, with or without modification,
are permitted provided that the following conditions are met:

* Redistributions of source code must retain the above copyright notice,
  this list of conditions and the following disclaimer.
* Redistributions in binary form must reproduce the above copyright notice,
  this list of conditions and the following disclaimer in the documentation
  and/or other materials provided with the distribution.
* Neither the name of the National Institute of Advanced Industrial Science
  and Technology (AIST) nor the names of its contributors may be used to 
  endorse or promote products derived from this software without specific prior 
  written permission.

THIS SOFTWARE IS PROVIDED BY THE COPYRIGHT HOLDERS AND CONTRIBUTORS "AS IS" 
AND ANY EXPRESS OR IMPLIED WARRANTIES, INCLUDING, BUT NOT LIMITED TO, 
THE IMPLIED WARRANTIES OF MERCHANTABILITY AND FITNESS FOR A PARTICULAR PURPOSE 
ARE DISCLAIMED. IN NO EVENT SHALL THE COPYRIGHT HOLDER OR CONTRIBUTORS BE 
LIABLE FOR ANY DIRECT, INDIRECT, INCIDENTAL, SPECIAL, EXEMPLARY, OR 
CONSEQUENTIAL DAMAGES (INCLUDING, BUT NOT LIMITED TO, PROCUREMENT OF SUBSTITUTE 
GOODS OR SERVICES; LOSS OF USE, DATA, OR PROFITS; OR BUSINESS INTERRUPTION) 
HOWEVER CAUSED AND ON ANY THEORY OF LIABILITY, WHETHER IN CONTRACT, STRICT 
LIABILITY, OR TORT (INCLUDING NEGLIGENCE OR OTHERWISE) ARISING IN ANY WAY OUT 
OF THE USE OF THIS SOFTWARE, EVEN IF ADVISED OF THE POSSIBILITY OF SUCH DAMAGE.
\end{verbatim}

Until version 8.25, Euslisp is distributed under following licence.

EusLisp can be obtained with its source code via ftp from
etlport.etl.go.jp (192.31.197.99).
Those who use EusLisp must observe  following articles
and submit a copy of license agreement (doc/LICENCE) to the author.
\begin{quote}
\begin{tabbing}
Toshihiro MATSUI \hspace{50mm} \= \\
Intelligent Systems Division, \>  \\
Electrotechnical Laboratory  \>  \\
1-1-4 Umezono,
Tsukuba, Ibaraki 3058568, JAPAN. \> email: matsui@etl.go.jp \\
\end{tabbing}
\end{quote}

Users are registered in the euslisp mailing list (euslisp@etl.go.jp),
where information for Q\&A, bug fix, and upgrade information is circulated.
This information has been accumulated in {\tt *eusdir*/doc/mails}.

\begin{enumerate}
\item The copyright of EusLisp belongs to
the author (Toshihiro Matsui) and Electrotechnical Laboratory.
The user must get agreement of use from the author.
\item Licensee may use EusLisp for any purpose other than military purpose.
\item EusLisp can be obtained freely from Elecrotechnical Laboratory
via ftp.
\item EusLisp may be copied or sold as long as articles described here
are observed. When it is sold, the seller must inform the customers
that the original EusLisp is free.
\item When licensees publicize their researches or studies which used EusLisp,
the use of EusLisp must be cited with appropriate bibliography.
\item Licensees may add changes to the source code of EusLisp.
The resulted program is still EusLisp as long as the
change does not exceed 50\% of codes,
and these articles must be observed for unchanged part.
\item The copyright of programs developped in EusLisp belongs to the
developper. However, he cannot extend his copyright over the main body
of EusLisp.
\item Neither the author nor ETL provides warranty.
\end{enumerate}

\subsection{Demonstrations}
Demonstration programs are found in {\tt demo} subdirectory.
{\tt cd} to {\tt *eusdir*} and run eusx.
\begin{description}
\item[Robot Animation] \index{animation}
Load {\tt demo/animdemo.l} from eusx.
Smooth animation of eta3 manipulator will be shown after a
precomputation of approximately 20 minutes.
\item[Ray-Tracing] \index{ray-tracing}
If you have 8-bit pseudo color display,
a ray-tracing image can be generated by loading
{\tt demo/renderdemo.l}.
Make sure {\tt geo/render.l} has already been compiled.
\item[Edge Vision]
Loading {\tt demo/edgedemo.l}, a sample gray-scale image is displayed.
You give parameters for choosing the gradient operator and 
edge thresholds.
Edges are found in a few second and overlayed on the original image.
\end{description}

\begin{figure}
% \epsfile{file=fig/eta3colavo.ps,width=150mm}
\mbox{
\epsfxsize=15cm
\epsfbox{fig/eta3colavo.ps}
}
\caption{Animation of Collision Avoidance Path Planning}
\end{figure}

\bibliography{/usr/users/matsui/biblio/lang,/usr/users/matsui/biblio/hirukawa}

\newpage
