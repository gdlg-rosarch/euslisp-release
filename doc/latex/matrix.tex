\newpage
\section{Geometric Functions}
\markright{\arabic{section}. Geometric Functions}

\subsection{Float-vectors}

A float-vector is a simple vector whose elements are specialized to
floating point numbers.
A float-vector can be of any size.
When {\em result} is specified in an argument list,
it should be a float-vector that holds the result.

\begin{refdesc}

\funcdesc{float-vector}{\&rest numbers}{
makes a new float-vector whose elements are {\em numbers}.
Note the difference between {\tt (float-vector 1 2 3)} and {\tt \#F(1 2 3)}.
While the former create a vector each time it is called, the latter
does when it is read.}

\funcdesc{float-vector-p}{obj}{
returns T if {\em obj} is a float-vector.}

\funcdesc{v+}{fltvec1 fltvec2 \&optional result}{
adds two float-vectors.}

\funcdesc{v-}{fltvec1 \&optional fltvec2  result}{
subtract float-vectors. If fltvec2 is omitted, fltvec1 is negated.}

\funcdesc{v.}{fltvec1 fltvec2}{computes the inner product of two float-vectors.}

\funcdesc{v*}{fltvec1 fltvec2 \&optional result}{
computes the outer product of two float-vectors.}

\funcdesc{v.*}{fltvec1 fltvec2 fltvec3}{
computes the scaler triple product [A,B,C]=(V. A (V* B C))=(V. (V* A B) C).}

\funcdesc{v$<$}{fltvec1 fltvec2}{
returns T if every element of fltvec1 is smaller  than
the corresponding element of fltvec2.}

\funcdesc{v$>$}{fltvec1 fltvec2}{
returns T if every element of fltvec1 is larger than
the corresponding element of fltvec2.}

\funcdesc{vmin}{\&rest fltvec}{
finds the smallest values for each dimension in {\em fltvec},
and makes a float-vector from the values.
{\bf Vmin} and {\bf vmax} are used to find the minimal bounding box
from coordinates of vertices.}

\funcdesc{vmax}{\&rest fltvec}{
finds the greatest values for each dimension in {\em fltvec},
and makes a float-vector from the values.}

\funcdesc{minimal-box}{v-list minvec maxvec [err]}{
computes the minimal bounding box for a given vertex-list,
and stores results in {\em minvec} and {\em maxvec}.
If a floating number err is specified, the minimal box is grown
by the ratio, i.e. if the err is 0.01, each element of minvec is decreased
by 1\% of the distance between minvec and maxvec,
and each element of maxvec is increased by 1\%.
{\bf Minimal-box} returns the distance between minvec and maxvec.}

\funcdesc{scale}{number fltvec \&optional result}{
the scaler {\em number} is multiplied to the every element of fltvec.}

\funcdesc{norm}{fltvec}{$|fltvec|$}
\funcdesc{norm2}{fltvec}{$|fltvec|^2=({\bf v.} fltvec fltvec)$}
\funcdesc{normalize-vector}{fltvec [result]}{
normalizes {\em fltvec} to have the norm 1.0.}
\funcdesc{distance}{fltvec1 fltvec2}{
returns the distance $|fltvec-fltvec2|$ between two float-vectors.}
\funcdesc{distance2}{fltvec1 fltvec2}{
$|fltvec-fltvec2|^2$}

\funcdesc{homo2normal}{homovec \&optional normalvec}{
A homogeneous vector {\em homovec} is converted to its normal representation.}

\funcdesc{homogenize}{normalvec \&optional homovec}{
A normal vector {\em normalvec} is converted to its homogenous representation.}

\funcdesc{midpoint}{p p1 p2 \&optional result}{
{\em P} is float, and {\em p1} and {\em p2} are float-vectors of the same
dimension.
A point $(1-p) p1 + p p2$, which is the point
that breaks {\em p1-p2} by the ratio $p:(1-p)$, is returned.}

\funcdesc{rotate-vector}{fltvec theta axis \&optional result}{
rotates 2D or 3D {\em fltvec} by {\em theta} radian around {\em axis}.
{\em Axis} can be one of {\em :x, :y, :z, 0, 1, 2} or NIL.
When {\em axis} is NIL, {\em fltvec} is taken to be two dimensional.
To rotate a vector around an arbitrary axis in 3D space, 
make a rotation matrix by the {\bf rotation-matrix} function and
multiply it to the vector.}

\end{refdesc}

\subsection{Matrix and Transformation}

A matrix is a two-dimensional array whose elements are all floats.
In most functions a matrix can be of any size,
but the {\bf v*, v.*}, {\bf Euler-angle} and {\bf rpy-angle} functions
can only handle three dimensional matrices.
{\bf Transform, m*} and {\bf transpose} do not restrict
the matrices to be square,
and they operate on general n*m size matrices.

Functions that can accept result parameter
places the computed result there, and no heap is wasted.
All matrix functions are intended for the transformation in the normal
coordinate systems, and not in the homogeneous coordinates.

The {\bf rpy-angle} function decomposes a rotation matrix into three components
of rotation angles around z, y and x axes of the world coordinates.
The {\bf Euler-angle} function is similar to {\bf rpy-angle} but
decomposes into rotation angles around local z, y and again z axes.
Both of these functions return two solutions since angles can be
taken in the opposite directions.

\begin{verbatim}
; Mat is a 3X3 rotation matrix.
(setq rots (rpy-angle mat))
(setq r (unit-matrix 3))
(rotate-matrix r (car rots) :x t r)
(rotate-matrix r (cadr rots) :y t r)
(rotate-matrix r (caddr rots) :z t r)
;--> resulted r is equivalent to mat
\end{verbatim}

To keep track of pairs of a position and a orientation in 3D space, use
the {\bf coordinates} and {\bf cascaded-coords} classes
detailed in the section \ref{Coordinates}.

\begin{refdesc}
\funcdesc{matrix}{\&rest elements}{
makes a new matrix from {\em elements}.
Row x Col = (number of elements) x (length of the 1st element).
Each of {\em elements} can be of any type of sequence.
Each sequence is lined up as a row vector in the matrix.}

\funcdesc{make-matrix}{rowsize columnsize \&optional init}{
makes a matrix of $rowsize \times columnsize$.}

\funcdesc{matrixp}{obj}{
T if {\em obj} is a matrix, i.e. {\em obj} is a two dimensional array
and its elements are floats.}

\funcdesc{matrix-row}{mat row-index}{
extracts a row vector out of matrix {\em mat}.
{\bf matrix-row} is also used to set a vector in a particular row
of a matrix using in conjunction with {\bf setf}.}
\funcdesc{matrix-column}{mat column-index}{
extracts a column vector out of {\em mat}.
{\bf matrix-column} is also used to set a vector in a particular
column of a matrix using in conjunction with {\bf setf}.}
\funcdesc{m*}{matrix1 matrix2 \&optional result}{
concatenates {\em matrix1} and {\em matrix2}.}
\funcdesc{transpose}{matrix \&optional result}{
transposes {\em matrix}, i.e. columns of {\em matrix} are exchanged with
{\em rows}.}
\funcdesc{unit-matrix}{dim}{
makes an identity matrix of {\em dim} $\times$ {\em dim}.}
\funcdesc{replace-matrix}{dest src}{
replaces all the elements of dest matrix with ones of src matrix.}

\funcdesc{scale-matrix}{scalar mat}{
multiplies {\em scaler} to all the elements of {\em mat}.}

\funcdesc{copy-matrix}{matrix}{
makes a copy of {\em matrix}.}

\funcdesc{transform}{matrix fltvector \&optional result}{
multiplies {\em matrix} to {\em fltvector} from the left.}
\funcdesc{transform}{fltvector matrix \&optional result}{
multiplies {\em matrix} to {\em fltvector} from the right.}

\funcdesc{rotate-matrix}{matrix theta axis \&optional world-p result}{
multiplies a rotation matrix from the left (when world-p is non-nil)
or from the right (when world-p is nil).
When a matrix is rotated by {\bf rotate-matrix},
the rotation axis {\bf :x, :y, :z} or 0,1,2
may be taken either in the world coordinates or in the local coordinates.
If {\em world-p} is specified nil, it means rotation along the
axis in the local coordinate system and the rotation matrix is multiplied
from the right.
Else if worldp is non-nil, the rotation is made in the
world coordinates and the rotation matrix is multiplied from the left.
If NIL is given to {\em axis}, {\em matrix} should be two dimensional and
the rotation is taken in 2D space where {\em world-p} does not make sense.}

\funcdesc{rotation-matrix}{theta axis \&optional result}{
makes a 2D or 3D rotation matrix around {\em axis} which can be any of
:x, :y, :z, 0, 1, 2, a 3D float-vector or NIL.
When you make a 2D rotation matrix, axis should be NIL.}

\funcdesc{rotation-angle}{rotation-matrix}{
extracts a equivalent rotation axis and angle from {\em rotation-matrix}
and a list of float and float-vector is returned.
NIL is returned when {\em rotation-matrix} is a unit-matrix.
Also if the rotation angle is too small, the result may have errors.
When {\em rotation-matrix} is 2D,  the single angle value is returned.}

\funcdesc{rpy-matrix}{ang-z ang-y ang-x}{
makes a rotation matrix defined by roll-pitch-yaw angles.
First, a unit-matrix is rotated by ang-x radian along X-axis.
Next, ang-y around Y-axis and finally ang-z around Z-axis.
All the rotation axes are taken in the world coordinates.}

\funcdesc{rpy-angle}{matrix}{
extracts two triplets of roll-pitch-yaw angles of {\em matrix}.}

\funcdesc{Euler-matrix}{ang-z ang-y ang2-z}{
makes a rotation matrix defined by three Euler angles.
First, a unit-matrix is rotated {\em ang-z} around Z-axis, next, {\em ang-y}
around Y-axis and finally {\em ang2-z} again around Z-axis.
All the rotation axes are taken in the local coordinates.}

\funcdesc{Euler-angle}{matrix}{
extracts two tuples of Euler angles.}

%\funcdesc{set-matrix-row}{mat row-index values}{
%puts values in the {\em row-index}th row of matrix {\em mat}.}

%\funcdesc{set-matrix-column}{mat column-index values}{
%put values in the {\em column-index}th column of of matrix {\em mat}.}
\end{refdesc}

\subsection{LU decomposition}
{\bf lu-decompose} and {\bf lu-solve} are provided to solve
simultaneous linear equations.
First, {\bf lu-decompose} decomposes a matrix into a lower triangle matrix
and an upper triable matrix.
If the given matrix is singular, {\bf LU-decompose} returns NIL, otherwise
it returns the permutation vector which should be supplied to {\bf LU-solve}.
{\bf Lu-solve} computes the solution for a LU matrix with a given constant vector.
This method is efficient if solutions for many combinations 
of different constant vectors and the same factor matrix are required.
%These functions are coded in C, and they computes a solution
%for a 3*3 matrix within 3 milli-sec on sun3.
{\bf Simultaneous-equation} would be  more handy when you wish to get
only one solution.
{\bf Lu-determinant} computes a determinant of a lu-decomposed matrix.
{\bf Inverse-matrix} function computes an inverse matrix using {\bf lu-decompose}
once, and {\bf lu-solve} n times.
Computation time for a 3*3 matrix is estimated to be 4 milli-sec.

\begin{refdesc}

\funcdesc{lu-decompose}{matrix \&optional result}{
performs lu-decomposition of {\em matrix}.}

\funcdesc{lu-solve}{lu-mat perm-vector bvector [result]}{
solves a linear simultaneous equations which has already been lu-decomposed.
{\em perm-vector} should be the result returned by {\bf lu-decompose}.}

\funcdesc{lu-determinant}{lu-mat perm-vector}{
computes the determinant value for a matrix which has already been 
lu-decomposed.}

\funcdesc{simultaneous-equation}{mat vec}{
solves a linear simultaneous equations whose coefficients are described in
{\em mat} and constant values in {\em vec}.}

\funcdesc{inverse-matrix}{mat}{
makes the inverse matrix of the square matrix, {\em mat}.}

\funcdesc{pseudo-inverse}{mat}{
computes the pseudo inverse matrix using the singular value decomposition.}

\end{refdesc}

\newpage

\subsection{\label{Coordinates}Coordinates}

Coordinate systems and their transformations are represented by
the {\bf coordinates} class.
Instead of 4*4 (homogeneous) matrix representation,
coordinate system in EusLisp is represented by a combination of
a 3*3 rotation matrix and a 3D position vector
mainly for speed and generality.

\begin{refdesc}

\classdesc{coordinates}{propertied-object}
{(pos :type float-vector \\
\>  rot :type array)}
{defines a coordinate system with a pair of a position vector and a 3x3
rotation matrix.}

\funcdesc{coordinates-p}{obj}{
returns T when obj is an instance of coordinates class or its subclasses.}

\methoddesc{:rot}{}{
returns the 3X3 rotation matrix of this coords.}

\methoddesc{:pos}{}{
returns the 3-D position vector of this coords.}

\methoddesc{:newcoords}{newrot \&optional newpos}{
updates the coords with newrot and newpos. Whenever a condition that changes
the state of this coords occurs,
this method should be called with the new rotation
matrix and the position vector.
This message may invoke another :update method to propagate the event.
If newpos is not given, newrot is given as a instance of coordinate class.
}

\methoddesc{:replace-coords}{newrot \&optional newpos}{
changes the rot and pos slots to be updated without calling newcoords method.
If newpos is not given, newrot is given as a instance of coordinate class.
}

\metdesc{:coords}{}

\methoddesc{:copy-coords}{\&optional dest}{
If dest is not given, :copy-coords makes another coordinates object
which has the same rot and pos slots. If dest is given, rot and pos of
this coordinates is copied to the dest coordinates.}

\methoddesc{:reset-coords}{}{
forces the rotation matrix of this coords to be identity matrix, and pos
vector to be all zero.}

\metdesc{:worldpos}{}
\metdesc{:worldrot}{}
\methoddesc{:worldcoords}{}{
Computes the position vector, the rotation matrix and the coordinates
of this object represented in the world coordinates. The coordinates class
is always assumed to be represented in world, these method can simply return
pos, rot and self.  These methods are provided for the compatibility with
cascaded-coords class which cannot be assumed to be represented in world.}

\methoddesc{:copy-worldcoords}{\&optional dest}{
First, worldcoords is computed, and it is copied to dest. If no dest
is specified, a coordinates object to store the result is newly created.}

\methoddesc{:rotate-vector}{vec}{
A vector is rotated by the rotation of this coords, i.e., an orientation
vector represented in this coords is converted to the representation
in the world.
The position of this coords does not affect rotation.}

\methoddesc{:transform-vector}{vec}{
A vector in this local coords is transformed to the representation in the world.}

\methoddesc{:inverse-transform-vector}{vec}{
A vector in the world is inversely transformed to the representation in this
local coordinate system.}

\methoddesc{:transform}{ trans \&optional (wrt :local)}{
Transform this coords by the trans represented in wrt coords.
Trans must be of type coordinates, and wrt must be one of keywords
{\tt :local, :parent, :world} or an instance of coordinates.
If wrt is {\tt :local}, the trans is applied from the right to this coords,
and if wrt is {\tt :world} or {\tt :parent},
the trans is multiplied from the left.
Else, if wrt is of type coordinates, the trans represented in the wrt
coords is first transformed to the representation in the world, and  it
is applied from the left.}

\methoddesc{:move-to}{trans \&optional (wrt :local)}{
Replaces the rot and pos of the coords with trans represented in wrt.}

\methoddesc{:translate}{ p \&optional (wrt :local)}{
changes the position of this object
relatively with respective to wrt coords.}

\methoddesc{:locate}{ p \&optional (wrt :local)}{
Changes the location of this coords with the parameter represented in wrt.
If wrt is {\tt :local}, then the effect is identical to 
{\bf :translate} with {\em wrt}={\tt :local}.}

\methoddesc{:rotate}{ theta axis \&optional (wrt :local)}{
Rotates this coords relatively by {\em theta} radian around the {\em axis}.
{\em Axis} is one of axis-keywords ({\tt :x, :y} and {\tt :z}) or an arbitrary float-vector.
{\em Axis} is considered to be represented in the {\em wrt} coords.
Thus, if {\em wrt}={\tt :local} and {\em axis}={\tt :z},
the coordinates is rotated around the z
axis of this local coords, and {\em wrt}={\tt :world} or {\tt :parent},
the coords is rotated around the z axis of world coords.
In other words, if {\em wrt}={\tt :local},
a rotation matrix is multiplied from the right of this coords,
and if {\em wrt}={\tt :world} or {\tt :parent},
a rotation matrix is multiplied from the left.
Note that even {\em wrt} is either {\tt :world} or {\tt :parent},
the {\em pos} vector of this coordinates does not change.
For the true rotation around the world axis,
an instance of coordinates class representing the
rotation should be given to {\bf :transform} method.}

\methoddesc{:orient}{ theta axis \&optional (wrt :local)}{
forces setting rot.
This is an absolute version of :{\bf rotate} method.}

\methoddesc{:inverse-transformation}{}{
makes a new coords that is inverse to self.}

\methoddesc{:transformation}{ coords (wrt :local)}{
makes the transformation (an instance of coordinates) between this coords
and the coords given as the argument.
If {\em wrt}={\tt :local},
the result is represented in local coords,
i.e., if the resulted transformation is given
as an argument to :transform with {\em wrt}={\tt :local},
this coords is transformed
to be identical with the coords.}

\methoddesc{:Euler}{ az1 ay az2}{
sets rot with Euler angles, that are,
rotation angles around z ({\em az1},
y ({\em ay}) and again z {\em az2} axis of
this local coordinates system.}

\methoddesc{:roll-pitch-yaw}{roll pitch yaw}{
sets rot with roll-pitch-yaw angles:
rotation angles around x ({\em yaw}), y ({\em pitch}) and z ({\em roll})
axes of the world coordinate system.}

\methoddesc{:4x4}{\&optional mat44}{
If a matrix of 4x4 is given as {\em mat44},
it is converted to coordinates representation with a 3x3 rotation matrix and
a 3D position vector.
If {\em mat44} is not given, this coordinates is converted to 4x4
matrix representation.}

\longdescription{:init}{\&key \= :pos \hspace{8mm} \= \#f(0 0 0)
 \`[method] \\
 \> :rot \> \#2f((1 0 0) (0 1 0) (0 0 1)) \\
 \> :rpy \> roll pitch yaw \\
 \> :Euler \> az ay az2 \\
 \> :axis \>  rotation-axis \\
 \> :angle \> rotation-angle \\
 \> :4X4   \> 4x4 matrix \\
 \> :coords \> another coordinates \\
 \> :properties \>  a list of (ind . value) pair \\
 \> :name \> name property \\}{
initializes this coordinates object and sets rot and pos.
The meaning of each keyword follows:
\begin{description}
\item[:dimension] 2 or 3 (default is 3)
\item[:pos] specifies a position vector (defaulted to \#f(0 0 0))
\item[:rot] specifies a rotation matrix (defaulted to a unit-matrix)
\item[:Euler] gives a sequence of three elements for Euler angles
\item[:rpy] gives a sequence of three elements for roll-pitch-yaw
\item[:axis] rotation axis (:x,:y,:z or an arbitrary float-vector)
\item[:angle] rotation angle (used with :axis)
\item[:wrt] where the rotation axis is taken (default is :local)
\item[:4X4] 4X4 matrix is used to specify both pos and rot
\item[:coords] copies pos and rot from coords
\item[:name] set :name property 
\end{description}
{\tt :Angle} can only be used in conjunction with the {\tt :axis}
that is determined in the {\tt :wrt} coordinates.
Without regard to {\tt :wrt}, {\tt :Euler} always specifies
the Euler angles, {\em az1, ay} and {\em az2}, 
defined in the local coordinates,
and {\tt :rpy} specifies the angles around
{\em z}, {\em y} and {\em x} axes of the world coordinates.
Two or more of {\tt :rot, :Euler, :rpy, :axis} and {\tt :4X4}
cannot be specified simultaneously, although no error is reported.
Sequences can be supplied to the {\tt :axis} and {\tt :angle} parameters,
which mean successive rotations around the given axes.
List of pairs of an attribute and its value can be given as {\tt :properties}
argument. These pairs are copied in the plist of this coordinates.}

\subsection{CascadedCoords}

\classdesc{cascaded-coords}{coordinates}
{(parent descendants worldcoords manager changed)}
{defines a linked coordinates. {\bf Cascaded-coords} is often abbreviated
as {\bfx cascoords}.}

\methoddesc{:inheritance}{ }{
returns the inheritance tree list describing all the descendants of the
cascoords.
If {\tt a} and {\tt b} are the direct descendants of this coords,
and {\tt c} is a descendant of {\tt a}, {\tt ((a (c)) (b))} is returned.}

\methoddesc{:assoc}{childcoords \&optional relative-coords}{
{\em childcoords} is associated to this cascoords as a descendant.
If childcoords has been already assoc'ed to some other cascoords, first
childcoords is dissoc'ed since each cascoords can have only one parent.
The orientation or location of childcoords in the world does not change.}

\methoddesc{:dissoc}{childcoords}{
dissociates (removes) {\em childcoords} from the descendants list of this coords.
The orientation or location of childcoords in the world does not change.}

\methoddesc{:changed}{}{
informs this coords that the coordinates of parent has changed,
and the re-computation of worldcoords is needed when it is requested later.}

\methoddesc{:update}{ }{
is called by the {\bf :worldcoords} method
to recompute the current worldcoord.}

\methoddesc{:worldcoords}{}{
returns a coordinates object which represents this coord in the world
by concatenating all the cascoords from the root to this coords.
The result is held in this object and reused later.
Thus, you should not modify the resulted coords.}

\methoddesc{:worldpos}{}{returns rot of this coordinates represented in
the world.}

\methoddesc{:worldrot}{}{returns pos of this coordinates represented in
the world.}

\methoddesc{:transform-vector}{vec}{
Regarding {\em vec} represented in this local coords,
transforms it to the representation in the world.}

\methoddesc{:inverse-transform-vector}{vec}{
{\em vec} represented in the world is inversely transformed into the
representation in this local coords.}

\methoddesc{:inverse-transformation}{}{
makes an instance of coordinates which represents inverse transformation
of this coord.}

\metdesc{:transform}{trans  \&optional (wrt :local)}
\metdesc{:translate}{fltvec \&optional (wrt :local)}
\metdesc{:locate}{fltvec \&optional (wrt :local)}
\metdesc{:rotate}{theta axis \&optional (wrt :local)}
\methoddesc{:orient}{theta axis \&optional (wrt :local)}{
Refer to the descriptions in class {\bf coordinates}.}

\fundesc{make-coords}{\&key :pos :rot :rpy :Euler :angle :axis :4X4 :coords :name}

\fundesc{make-cascoords}{\&key :pos :rot :rpy :Euler :angle :axis :4X4 :coords :name}
\fundesc{coords}{\&key :pos :rot :rpy :Euler :angle :axis :4X4 :coords :name}
\funcdesc{cascoords}{\&key :pos :rot :rpy :Euler :angle :axis
:4X4 :coords :name}{
All these functions make new coordinates or cascaded-coordinates.
For the keyword parameter, see {\bf :init} method of class coordinates.}
\funcdesc{transform-coords}{coords1 coords2 \&optional (coords3 (coords))}{
Coords1 is applied (multiplied) to the coords2 from the left.
The product is stored in coords3.}
\funcdesc{transform-coords*}{\&rest coords}{
concatenates transformations listed in {\em coords}.
An instance of coordinates that represents the concatenated transformation
is returned.}
\funcdesc{wrt}{coords vec}{
transforms {\em vec} into the representation in {\em coords}.
The result is equivalent to
{\tt (send {\em coords} :trans\-form-\-vec\-tor {\em vec})}.}
\end{refdesc}

\newpage
\subsection{Relationship between transformation matrix and coordinates class}

Relationship between transformation matrix and coordinates class is described,
where a transformation matrix T represents a $4\times4$(homogeneous) matrix as below.
\[
T = \begin{pmatrix}
\mathbf{R}_T & \mathbf{p}_T \\
\mathbf{0} & 1
\end{pmatrix}
\]

$\mathbf{R}_T$ is a $3\times3$ matrix, and $\mathbf{p}_T$ is a $3\times1$ matrix
(a float-vector which has 3 elements in euslisp).
Coordinates class has
slot variables rot and pos. They are $\mathbf{R}_T$ and $\mathbf{p}_T$ respectively.

\subsection*{Getter method for rotation matrix and position}

$\mathbf{R}$ and $\mathbf{p}$ can be obtained using methods of the coordinates class.

T is an instance of the coordinate class.

\begin{refdesc}

{\bf (send T :rot)}
\desclist{\hspace{0mm}
$\Rightarrow$ $\mathbf{R}_T$
}

{\bf (send T :pos)}
\desclist{\hspace{0mm}
$\Rightarrow$ $\mathbf{p}_T$
}

\end{refdesc}

\subsubsection*{Methods for transforming vectors}

$\mathbf{v}$ is 3-D position vector.

\begin{refdesc}

{\bf (send T :rotate-vector $\mathbf{v}$)}
\desclist{\hspace{0mm}
$\Rightarrow$  $\mathbf{R}_T \mathbf{v}$
}

{\bf (send T :inverse-rotate-vector $\mathbf{v}$)}
\desclist{\hspace{0mm}
$\Rightarrow$ $\mathbf{v}^T \mathbf{R}_T$
}

{\bf (send T :transform-vector $\mathbf{v}$)}
\desclist{\hspace{0mm}
$\Rightarrow$ $\mathbf{R}_T\mathbf{v} + \mathbf{p}_T$

Converts a vector represented in a local coordinate system T
to a vector represented in the world coordinate system.
}

{\bf (send T :inverse-transform-vector $\mathbf{v}$)}
\desclist{\hspace{0mm}
$\Rightarrow$ $\mathbf{R}_T^{-1}\left( \mathbf{v} - \mathbf{p}_T \right)$

Converts a vector represented in the world coordinate system.
to a vector represented in a local coordinate system T.
}

\end{refdesc}

\subsubsection*{Methods returing coordinates without modifying itself}

\begin{refdesc}

{\bf (send T :inverse-transformation)}
\desclist{\hspace{0mm}
$\Rightarrow$ $T^{-1}$

Returns inverse matrix.
\[
T^{-1} = \begin{pmatrix}
 \mathbf{R}_T^{-1} & -\mathbf{R}_T^{-1}\mathbf{p}_T \\
 \mathbf{0} & 1
\end{pmatrix}
\]
}

{\bf (send T :transformation A (\&optional (wrt :local)))}
\desclist{\hspace{0mm}
when wrt == :local, $T^{-1}A$ is returned.

when wrt == :world, $AT^{-1}$ is returned.

when wrt == W (coordinates class), $W^{-1}AT^{-1}W$ is returned.
}

\end{refdesc}

\subsubsection*{Methods modifying itself}

A is an instance of the coordinates class.

$\Leftrightarrow$ represents that
slot variables (pos or rot) refer to a given instance (matrix or float vector).
Please note that when one is changed, the other also reflects the change.

$\leftarrow$ represents substitution.

\begin{refdesc}

{\bf (send T :newcoords A)}
\desclist{\hspace{0mm}
$\mathbf{R}_T \Leftrightarrow \mathbf{R}_A$

$\mathbf{p}_T \Leftrightarrow \mathbf{p}_A$
}

{\bf (send T :newcoords $\mathbf{R}$ $\mathbf{p}$)}
\desclist{\hspace{0mm}
$\mathbf{R}_T \Leftrightarrow \mathbf{R}$

$\mathbf{p}_T \Leftrightarrow \mathbf{p}$
}

{\bf (send T :move-to A (\&optional (wrt :local)))}
\desclist{\hspace{0mm}
when wrt == :local,$T \leftarrow TA$

when wrt == :world,$T \Leftrightarrow A$

when wrt == W (coordinates class), $T \leftarrow WA$
}

{\bf (send T :translate $\mathbf{v}$ (\&optional (wrt :local)))}
\desclist{\hspace{0mm}
when wrt == :local,
$\mathbf{p}_{T} \leftarrow \mathbf{p}_{T} + \mathbf{R}_{T}\mathbf{v}$

when wrt == :world,
$\mathbf{p}_{T} \leftarrow \mathbf{p}_{T} + \mathbf{v}$

when wrt == W (coordinates class),
$\mathbf{p}_{T} \leftarrow \mathbf{p}_{T} + \mathbf{R}_{W}\mathbf{v}$
}

{\bf (send T :locate $\mathbf{v}$ (\&optional (wrt :local)))}
\desclist{\hspace{0mm}
when wrt == :local,
$\mathbf{p}_{T} \leftarrow \mathbf{p}_{T} + \mathbf{R}_{T} \mathbf{v}$

when wrt == :world,
$\mathbf{p}_{T} \leftarrow \mathbf{v}$

when wrt == W (coordinates class),
$\mathbf{p}_{T} \leftarrow \mathbf{p}_{W} + \mathbf{R}_{W}\mathbf{v}$
}

{\bf (send T :transform A (\&optional (wrt :local)))}
\desclist{\hspace{0mm}
when wrt == :local, $T \leftarrow TA$

when wrt == :world, $T \leftarrow AT$

when wrt == W (coordinates class), $T \leftarrow$ $\left( WAW \right)^{-1} T$
}

\end{refdesc}

\newpage

