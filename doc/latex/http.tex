
\section{HTTP}

\subsection{HTTP Client}

\begin{refdesc}

\classdesc{URL-pathname}{pathname}{server port protocol}{
extends pathname to allow URL notation.}

\funcdesc{url-pathname}{name}{
instantiates url-pathname class object from url string or url-pathname class object.}

\funcdesc{escape-url}{url \&optional (ss *standard-output*) (queryp t)}{
writes percent-escaped {\tt url} to stream {\tt ss} (default: {\tt *standard-output*}).
If {\tt queryp} is T, then {\tt Space} in {\tt url} is encoded to {\tt +}, otherwise escaped as {\tt Space}.
This option is convenient for sending url query to server with separation.}

\funcdesc{escaped-url-string-from-namestring}{url-string \&optional (queryp t)}{
returns result of {\tt escape-url} as string.}

\funcdesc{unescape-url}{url \&optional (ss *standard-output*) (queryp t)}{
unescapes percent-escaped {\tt url} and writes unescaped url to stream {\tt ss}.}

\funcdesc{unescaped-url-string-from-namestring}{url-string \&optional (queryp t)}{
returns result of {\tt unescape-url} as string.}

\funcdesc{read-http}{url \&key (timeout 10) (retry 5)}
{makes a socket connection to the designated url, and
read the html document.
The result is a list of tags and plain strings.
HTML tags are converted as lists consisting of the tag-name
and argument lists.
For example, the following html document,
results in the following list.
Note that tags are represented as lists, in which
the directive is represented as a symbol
followed by symbols or strings.
Whether an argument is represented as symbol or string
reflects how the original argument is described.}


\begin{verbatim}
<body bgcolor=#ffa080>
<h1> EusLisp Title</h1>
<li> item1 </li>
<a href="http://www.etl.go.jp/~matsui/eus/euslisp.html"> euslisp</a>
</body>
\end{verbatim}

\begin{verbatim}
("HTTP/1.1 200 OK"
 "Date: Sun, 21 May 2000 11:47:00 GMT"
 "Server: Apache/1.3.9 (Unix)"
 "Last-Modified: Sun, 21 May 2000 11:19:35 GMT"
 "ETag: \"4f014-c7-3927c647\""
 "Accept-Ranges: bytes"
 "Content-Length: 199"
 "Content-Type: text/html"
 (head) (title) " Toshihiro Matsui on t570" (/title) (/head)
 (body bgcolor |#FFA080|)
 (h1) " Title Line" (/h1)
 (li) " item1 " (/li)
 (a href "http://www.etl.go.jp/~matsui/eus/euslisp.html")
 " euslisp"
 (/a)
 (/body))
\end{verbatim}

\funcdesc{extract-html}{tag html-list}{
returns a list of strings (and tags) sandwitched by tag and /tag.}

\funcdesc{remove-html-tags}{html-list}{
removes tags from the html-list leaving only texts (strings).}

\end{refdesc}

\subsection{HTTP CGI Programming}

EusLisp can be used for  CGI programming.
The following is a typical cgi entry to a EusLisp program.
This code piece should be placed under .../cgi-bin/ or under
any directories where ExecCGI is allowed.  The code piece
must have execute permission by the `nobody' user.
Note that CGI programs are executed by httpd whose owner is
nobody. You also have to set up some environment variables in the
code piece, for nobody does not know anything particular for EusLisp.

\begin{verbatim}
#! /bin/csh
setenv EUSDIR /usr/local/eus/
setenv LD_LIBRARY_PATH /usr/local/eus/Linux/lib
/usr/local/bin/eus /usr/local/eus/lib/demo/mycgi.l
\end{verbatim}

mycgi.l is a lisp source program, which should load
"\$EUSDIR/lib/llib/httpcgi.l" at the beginning.
The CGI program is responsible for obtaining CGI arguments,
generating an html header, and producing html contents.
The arguments are obtained by the {\tt get-cgi-query} function,
and split to a list by the {\tt parse-cgi-query} function.
The parsed list contains pairs of argument-name and argument-value.
For example, if the CGI is invoked by href to
"/cgi-bin/eus.cgi?user=matsui\&age=43",
the parsed list gives ((user matsui) (age 43)).

All normal CGI output should go to *cgi-out*.
Before any html document, a header should be generated
by the {\tt html-header} function.
If there is any error message written to *error-output*,
it appears in the httpd's error-log.
When the work is done and html document finishes by '</html> tag,
the process may close the connection (*cgi-out*) and may exit.
Normal exit of the CGI process usually signals the httpd to
send the data to http clients.


{*cgi-out*} is the output stream to which the generated html document
should be sent.

\begin{refdesc}

\funcdesc{gen}{string}{ Outputs the string to *cgi-out* stream,
which is then forwarded to to the client (browser).}

\funcdesc{html}{args ...}{generates args as one string.}

\funcdesc{html-table}{lst \&key heading (table-option "")}{
generates an html table.}

\funcdesc{get-cgi-query}{}{gets the argument to this CGI program.
First, the REQUEST\_METHOD environment variable is looked up, and
the POST/GET method is determined. The query string is obtained
from the QUERY\_STRING environment variable or from the standard input.
Anyways, the result is returned in one string.}

\funcdesc{parse-http-query}{query-string}

\funcdesc{html-header}{}{generates the html header, 
usually a simple string of two lines,
"Content-type: text/html\~\%\~\%".}

\funcdesc{qval}{arg query}{
arg (symbol) is searched in the query list,
and the value is returned if found.
The result is converted to euc encoding from sjis encoding.}

\end{refdesc}

\subsection{Fast-CGI}

Whereas CGI is a convenient method to produce dynamic document on
the server side, it is not the very best choice due to a performance reason:
the cgi process must be spawned everytime a request arrives, and the
process invocation time is not always negligible.  In my measurement, 
the simplest CGI written in EusLisp needs 0.3 sec to respond.
In this sense, EusLisp or any other programming system with rich runtime
modules is not a very good choice for CGI writing.

Since this invocation load is a common problem for all CGI programs,
there is a clever work around called Fast-CGI.
The basic idea of the Fast-CGI is to allow
CGI processes to keep alive even one CGI request is fulfilled.
The httpd process communicates with a fast-cgi process via a TCP connection.
 
\begin{refdesc}
\classdesc{fcgi-connection}{propertied-object}{cookie host} {}

\macrodesc{fcgi-loop}{\&rest forms}{
repeats evaluation of {\it forms} each time http connection request
is accepted.}


\end{refdesc}

