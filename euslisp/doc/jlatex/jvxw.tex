\subsection{VxWorks}
\markright{\arabic{section}. VxWorks}
ホストとVxWorksとの通信機能が"vxworks/vxweus.l"ファイルで提供されている。
VxWorks上に{\bf vxwserv}サーバを常駐させることにより、
ホスト上のEusLispから{\bf vxwserv}に
コネクションを張り、vxwsプロトコルに従った
コマンドを送ることにより、VxWorksの関数を起動し、引数を送り、
結果を受け取ることができる。

VxWorksのソフトはSunのcコンパイラによって開発することができる上、
データ表現がsun3, sun4, VME147の間で共通であることを利用して、
vxwsプロトコルは、バイナリモードで動作することができる。

\subsubsection{VxWorks側の起動}
VxWorksにログインし、"*eusdir*/vxworks/vxwserv.o"をロードする。
その後、{\bf vxwserv}タスクをspawnする。
{\bf vxwserv}はVxWorks上の2200番ポートをlistenする。
2200が塞がっている場合、2201, 2202, ...を試す。
正しくbindされたポート番号が表示される。

\begin {verbatim}
% rlogin asvx0  (あるいはetlic2上であれば、% tip asvx[01] も可能)
-> cd "atom:/usr/share/src/eus/vxworks"
-> ld <vxwserv.o
-> sp vxwserv
port 2200 is bound.
\end{verbatim}

VxWorksのiコマンドで、{\bf vxwserv}タスクが常駐したことを確かめる。
同じ要領で、eusから呼び出したいVxWorksのプログラムをVxWorks上にロードする。
その後、EuslispとVxWorksとの
コネクションが張られると、{\bf vxwserv}を走らせたTTYに、次のようなメッセージが
出力される。

\begin{verbatim}
CLIENT accepted: sock=9 port = 1129: family = 2: addr = c01fcc10:
VxWserv started with 16394 byte buffer
\end{verbatim}

\subsubsection{ホスト側の起動}
任意のマシンの上でeusを起動し、"vxworks/vxweus"をロードする。
{\bf connect-vxw}関数を用いて{\bf vxwserv}に接続する。
接続後、ソケットストリームが{\bf *vxw-stream*}にバインドされる。
以下に、コネクトの例を示す。この例では、VxWorks上のsin, vadd関数を
euslispの関数VSIN,VADとして定義している。


\begin {verbatim}
(load "vxworks/vxweus")
(setq s (connect-vxw :host "asvx0" :port 2200 :buffer-size 1024))
(defvxw VSIN "_sin" (theta) :float)
(defvxw VAD "_vadd" (v1 v2) (float-vector 3))
\end{verbatim}

VxWorks上に作成される関数が、vxwsを通じて呼び出されるためには、
次の条件を満たさなければならない。

\begin{enumerate}
\setlength{\itemsep}{0cm}
\item 引数は、32個以内であること、引数に受け取るベクタの容量の合計が
connect-vxwの:buffer-sizeで指定した値を越えないこと
\item structを引数にしないこと、必ずstructへのポインタを引数にすること
\item 結果は、int, float, doubleまたは、それらの配列のアドレスであること
\item 配列のアドレスを結果とする場合、その配列の実体は、
関数の外部に取られていること
\end{enumerate}

\begin{refdesc}

\longdescription{connect-vxw}{\&key \= (:host "asvx0") \` [関数] \\
\> (:port 2200)\\
\> (:buffer-size 16384)\\
\> (:priority 1280)\\
\> (:option \#x1c)}{
{\em :host}に対してvxwsプロトコルによる通信のためのソケットストリームを作成し、そのストリームを返す。
{\em :host}には、ネットワークにおけるVxWorksのアクセス番号あるいはアクセス名を指定する。
{\em :port}には、VxWorks上の{\bf vxwserv}がバインドしたポートを捜すための最初のポート番号を指定する。
このポート番号から、増加方向に接続を試行する。
{\em :option}のコードについては、VxWorksの、spawn関数を参照のこと。
コネクションは、同時に複数張ってよい。}

\funcdesc{vxw}{vxw-stream entry result-type args}{
{\bf vxw}は、{\em vxw-stream}に接続されているVxWorksの関数{\em entry}を呼び出し、
その関数に引き数{\em args}を与えて{\em result-type}で指定された結果を得る。
{\em vxw-stream}には、{\bf connect-vxw}で作成したソケットストリームを与える。
{\em entry}には、VxWorksの関数名をストリングで指定するか、あるいは関数の
アドレスを整数で指定する。
関数のアドレスを知るには、VxWorksのfindsymbolを呼び出す。
知りたいシンボルは、通常、"\_"で始まることに注意。
{\em entry}がストリングの場合、VxWorks上でシンボルテーブルの逐次探索が行われる。
{\em result-type}には、結果のデータ型({\tt :integer}または{\tt :float})、
あるいはデータを受け取るベクタ型を指定する。
ベクタは、{\tt float-vector, integer-vector, string}のインスタンスである。
{\tt general vector}(lispの任意のオブジェクトを要素とするベクタ)は指定できない。
結果型は、必ず、実際のVxWorks関数の結果型と一致しなければならない。
{\em args}には、{\em entry}に与える引き数を指定する。
引数に許されるEusLispデータは、{\tt integer, float, string, integer-vector,
float-vector, integer-matrix, float-matrix}である。
ポインタを含んだ一般のオブジェクト、一般のベクトルは送れない。
また、送られたベクトルデータは、一旦{\bf vxwserv}が獲得したバッファの中に蓄積される。
例えば、VxWorksに定義された関数"sin"を呼び出すためには、次のように実行すればよい。
{\tt (vxw *vxw-stream* "sin" :float 1.0)}
}

\macrodesc{defvxw}{eus-func-name entry args \&optional (result-type :integer)}{
{\bf defvxw}は、findsymbolを用いて{\bf vxw}を呼び出して、VxWorksの関数の高速な呼び出しを
実現するためのマクロである。
VxWorksの関数{\em entry}を呼び出すためのEuslispの関数{\em eus-func-name}を定義する。
このマクロを実行後は、{\em eus-func-name}を呼び出すことにより、VxWorksの関数を呼び出すことができる。
このとき、呼び出しに使用されるソケットストリームは{\bf *vxw-stream*}に固定されている。
ただし、VxWorks側で、関数をコンパイルし直して再ロードした場合、
新しい関数定義が呼ばれるようにするためには、eus側で、{\bf defvxw}をもう一度
実行し直して、最新のエントリアドレスが指定されるようにする必要がある。
}
\end{refdesc}
\newpage
